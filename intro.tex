\documentclass[10pt,twocolumn]{article}
\usepackage{times}
\begin{document}
\title{Artificial Life Creation Comparing Genetic Algorithms And Genetic Programming}
\author{Tom Cammann\\\\
Computer Science, cy004947@reading.ac.uk}
\date{}
\maketitle
\begin{abstract}

	Abstract stuff here here here
\end{abstract}


\section{INTRODUCTION}

Artificial life or ALife is the study and implementation of systems that are simulations or replications of natural life.
These systems can be biochemical, mechanical, computer software and many other medium that supports the existence of structured self controlled information.
This report will be looking at software artificial life, or Soft ALife and how it can be implemented using evolutionary computation.
Evolutionary computation has been used for the creation of artificial life for many decades, REF + EXAMPLE.
Using evolutionary computation has many qualities that replicate naturally evolved life, and is one few computational methods known that can generate solutions to problems without human interaction.
Evolutionary computation has been applied to artificial life to generate life forms that can inhabit an environment and are 'solutions' to this environment.

\paragraph{}
Evolutionary computation generally works by generating a population of solutions to a problem, and then using this population to create a better next population that can solve the given problem better.
To create a better population crossover, mutation and selection are used in various forms to generate a fitter population.
Each of these populations is referred to as a generation.
Each member in the population is a solution to the problem, however these solutions do not have to be correct, and are often close to correct before many generations have passed.
Each member of the population is assigned a fitness value to correspond with how close the solution is to correct, or how effective the solution is.
In the Context of artificial life this fitness value may correspond to how well the life form survives in a given environment.

\paragraph{}

\section{GENETIC ALGORITHMS}

Genetic algorithms were first designed and implemented by ??? in ??? when ???.

The general form of a genetic algorithm uses genes to represent parameters inside a program.
These parameters effect the running of a program.
These genes were historically represented in binary form, however more modern implementations can use any value as a gene, from programming objects to double floating point numbers.
These genes must be able to mutate, this is easy to understand when using binary number, to mutate a binary number you flip one bit in the gene sequence. 

\subsection{POPULATION REPRESENTATION}
In genetic algorithms each member of the population (or candidate solution) is represented by a list or string of values. Each value represents a parameter that will be used in the solution. The value could represent the number of iterations of a loop or represent 

\subsection{MUTATION}
In genetic algorithms mutation occurs on a per gene basis. If a mutation for a population member is needed then a gene in that 


\section{METHODOLOGY}

\subsection{GENETIC ALGORITHMS}


\end{document}
