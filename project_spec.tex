\documentclass[12]{article}
\usepackage[pdftex]{graphicx}
\graphicspath{{./}{figs/}}
\begin{document}
\title{Appendix 4 - Further Particular for ‘CS’ degree project}
\maketitle
\newcounter{counter}
\section{}
Most Computer Science projects falls into three broad types. 
\begin{list}{Type \Roman{counter}:~}{\usecounter{counter}\setlength\labelwidth{3in}}
\item Projects which involve substantial design effort followed by an implementation
and testing phase. For example a compiler for a small subset of a certain programming 
language. A part of such projects will be in the form of a users manual, specifying what 
the package does and how it is to be used, together with support documentation giving 
sufficient details for someone, who so desires, to modify the coding. 
\item Projects in which the work of analysis and design is so complex as to constitute 
a complete project in itself. The end-project will usually be a system design document, 
but the project will show in exceptional detail full use of all the phases of the software 
lifecycle that precede and lead to the creation of  that design. As always, a critical 
assessment of the methods used will be expected in the report. 
\item Projects which involve substantial  analysis followed by a specification (and 
possibly a verification) phase. Typically these are projects of a theoretical nature which 
usually contain a high mathematical content. Such project may involve formal 
specifications, development of proof methods and refinement techniques. A prototype 
may be implemented for evaluation purposes or software written to provide automated 
tools. 
\end{list}
All reports should be organized into sections, and sections with a common theme should be 
arranged into chapters. Bearing in mind the different requirements of  each type of project, a 
possible report structure is shown below (the headings are meant to suggest themes and should 
not necessarily be taken literally). 
\subsection{Abstract, Contents page}
All reports should contain:  
\begin{list}{\roman{counter}}{\usecounter{counter}}
\item A title page consisting of - University of Reading 
School of Systems Engineering 
CS3Q2 / SE3Q2 - Computer Science Final Project Report  
Title of Project  
Name of Student  26 
Names of Supervisor, Second Reader 
Date  
\item An abstract page. The abstract should be a single page summary of what the project is about and to 
what extent its aims have been realized. Include all the important points and place as much information 
here as space permits. 
\item A contents page.  
\end{list}
\subsection{Introducing the Project and Specification}
This applies to all types of projects 
\begin{list}{\roman{counter}}{\usecounter{counter}}
\item Declaration of the aims and objectives of the project. 
\item Review of published work where appropriate.  General descriptions of the survey area and its 
importance/interest to the computing community. 
\item General statement of the problem, in other words, a general informal requirements specification.  
\item Division of responsibility if joint project.
\end{list}
\subsection{Analysis and Design}
\begin{list}{\roman{counter}}{\usecounter{counter}}
\item An in-depth analysis of the problem. Identification of the key problems associated with your project.  
\item This section will include a complete requirements analysis, transformed into a complete 
specification, transformed into a complete design. In the case of large projects the report will probably 
include only a selection of parts of the system. The selection should be made for clear, stated reasons, 
for example, that they illustrate particular difficulties experienced or specially complex parts of the 
problem domain. For type III projects this section will contain a discussion of the formal technique 
used including an explanation of notation and methods. 
\item Any other aspects of the design process. Describe any problems.  
\end{list}
\subsection{Development and Implementation}
\begin{list}{\roman{counter}}{\usecounter{counter}}
\item List the technical difficulties experienced; description and clearly labelled diagrams of the data 
structures employed and why they were chosen; high level coherently organized descriptions of the 
various component programs and how they relate together. The presentation and description of the 
underlying algorithms (but don’t simply duplicate listings in the text) 
\item Specifications of programs, detailed input  data and formats, output results layout, operating 
environment including compilers, library routines, etc.; The flow of control through the entire program 
or design should be indicated on a single side. Relationship between specifications and program listings 
should be made clear. 
\item For type III projects this section will contain a description of the system, details of the construction 
of a formal model of the system that is being studied. Where some form of prototyping has been 
attempted, its aims and usefulness must be discussed.  
\end{list}
\subsection{Results}
\begin{list}{\roman{counter}}{\usecounter{counter}}
\item Test data and the results produced, together with their significance. Use graphs and tables. 
Interpretation of results. This section may not apply to all projects.  
\end{list}
\subsection{Testing}
\begin{list}{\roman{counter}}{\usecounter{counter}}
\subsubsection{Testing and Quality (Type I and II projects)}
\item Details of strategy used for testing purposes.  Exhaustive testing is not expected, but the testing 
method used should be clearly shown. Why was a particular testing strategy chosen and what were the 
aims? Specify the steps used to ensure that quality is achieved. Did testing detect any problems? 
Results of testing. For type I projects this includes details of data vetting and error messages. For type 
II projects there needs to be evidence that the design as such has been verified in some industrystandard manner. Describe any problems in testing and ensuring quality.  27 
\subsubsection{Correctness Analysis (Type III projects)}
\item What can be deduced about the proposed design or system by reasoning about the formal model 
constructed? What correctness criteria can be formulated and can these correctness criteria be satisfied? 
Describe any problems in testing and ensuring quality. 
\end{list}
\subsection{Summary, Conclusions and Critique}
\begin{list}{\roman{counter}}{\usecounter{counter}}
\item Costings: In the preliminary report, costing techniques (as taught in Software Engineering) were 
used to produce a estimate of the time and effort required for the project. How would you reconcile the 
estimated costings with the actual cost? You will not be penalised if the estimated cost and the actual 
cost do not match. For type III projects, in place of the costing techniques used in Software 
Engineering, the student should try to quantify and discuss the use of formal techniques, not discarding 
the ideas of Software Engineering, but working out how they may be applied (or not) in a new context. 
\item To what extent have the aims of the project been fulfilled? Include a critical assessment of he design 
tools and methods used. Summarise the major problems and drawbacks with your approach. What are 
the main conclusions and findings of the project? 
\item How would you alter the method or approach in the light of experience? What else would you have 
liked to do if the time were available? Suggest possible improvements and areas for future work.
\end{list}
\end{document}
